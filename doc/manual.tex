\documentclass[12pt,letterpaper]{article}

\usepackage{keystroke}

\newcommand{\svdimage}{\texttt{SvdImage }}

\newcommand{\codeitem}[2]{\item[\texttt{\%> #1}] \hfill \\ #2}

\begin{document}

\title{SvdImage Manual}
\author{Tim Martin}
\date{Spring 2012}
\maketitle

\section{Description}
In an abstracted form, images are matrices of color values. Therefore, we can
perform linear algebra operations on images just as we do on general matrices.
Specifically, when we perform a Singular Value Decomposition (SVD) on an image,
we can represent it as the product of 3 matrices: the left-singular vectors,
a diagonal matrix of the decreasing singular values and the right-singular vectors.

\begin{equation}
A=USV^{T}
\end{equation}

Performing an SVD gets interesting for images when we truncate the factor
matrices. Truncation in this sense is the removal of a certain number of
trailing columns of $U$ and $V$ and trailing columns and rows of $S$. By
multiplying the matrix back into $A$ after truncating, we get back a similar
image to the original. Depending on the amount of truncation, we can keep a
``good'' representation of the image in less space! 

\svdimage is a program and developer library that performs a truncated SVD. It
is written in the Ruby programming language and provides a command line
interface.

\section{Features}
\begin{itemize}
\item Command Line Interface
\item Grayscale, RGB, and CMYK channel processing
\item Level of truncation can be ``automatic'' or user-specified
\item Supports most image file formats
\item Codec provides targeted file format (.svd)
\item Ruby Gem distribution
\item Can be used a library
\end{itemize}

\section{Compression}
The main application of \svdimage is to compress images. However, this does not
come for free. We are indeed losing information about the original image. This
happens by reducing the rank our image to some value. For example, an image
\texttt{dog.jpg}may have 200 linearly independent rows and columns. If we run
\svdimage on this image with a k value of 150, the output will be an image that
looks similar to \texttt{dog.jpg}, but is acutally now comprised of 150 linearly
independent rows and columns. In other words, 50 rows and columns will be the
linear combination of the other 150 rows and columns. 

Another caveat is that you may not always notice compression in file size. This
is dependent on a few factors (image content, colorspace, file format). Not each
file format stores truncated SVDs in a way optimal to this operation. I can't
imagine a file format that would be able to understand if some rows or columns
are linearly dependent.

However, this is not to say you will never see compression! Personally, most
times, I have.

\section{Sigma Ratio Threshold}
\svdimage comes with the option to use the \texttt{-a}/\texttt{--auto-k} flags.
These tell \svdimage that you would like to try to find a good truncation
automatically. This truncation is unique to to the image's singular values and
the sigma threshold. First, we set our sigma threshold equal to some value. By
default, \svdimage uses 0.2, but you may also specify this yourself by providing
an argument to the \texttt{-a}/\texttt{--auto-k} flags. Then, starting with a
$k=1$, we iterate through all $k$ until:

\begin{equation}
\mbox{sigma\_threshold} \geq \frac{\sum_{n=k+1}^{t} \sqrt{\sigma_{n}^{2}}}{\sum_{n=1}^{t} \sqrt{\sigma_{n}^{2}}}
\end{equation}

Intuitively, you may think of this as leaving out $\mbox{sigma\_threshold}\%$ of
the image's information, but this is not exactly correct.

\section{Question}
The project poses a question:
\begin{quote}
To get a good easily recognizable image, do you need to have $\sigma_{k+1}$ small
compares to 255, or just small compared to $\sigma_1$, or is there some other criterion of smallness
that is even more relevant?
Try this compression with portraits of faces. How small can you make k, and still
keep the portrait recognizable? In other words (roughly), what is the rank of a human
face?
\end{quote}

I believe I have answered this question with my explanation of the sigma ratio threshold. A good $k$ is not
dependent on 255 or just being small, but it is instead dependent on the
complexity of the image. $k$ must be chosen such that you do not lose too much
information from the image.

\section{Requirements}
\subsection{System}
\begin{itemize}
\item Ruby 1.9.2
\item GNU Scientific Library 1.15
\item ImageMagick 6.7.6-0
\item RubyGems 1.8.17
\end{itemize}
\subsection{Ruby Gems}
\begin{itemize}
\item rake
\item bundler
\end{itemize}
Bundler will be used to easily download and install other gems.

\section{Installation}
Once all requirements have been met, you can install \svdimage. If you are
unable to get these requirements or perform the following installation, please
contact me!
\begin{description}
\codeitem{tar -xzvf svdimage-0.1.0.tar.gz}{Extract the archive contents.}
\codeitem{cd svdimage-0.1.0/}{Change into the \svdimage directory.}
\codeitem{rake install}{Install \svdimage to your system.}
\end{description}
And that's it! \svdimage should be in your \texttt{\$PATH} and you can start
using it anywhere

\section{Usage}
\footnotesize
\begin{verbatim}
Usage: svdimage INPUT_IMAGE OUTPUT_IMAGE [OPTIONS]
    -c, --colorspace COLORSPACE      Defines the colorspace of output-image.
                                       Must be "rgb", "gray", or "cmyk".
                                       Defaults to "rgb".
    -k RANK                          Truncates the SVD to rank RANK. May not be
                                       used with -a/--auto-k
    -a, --auto-k [SIGMA_THRESHOLD]   Truncates the SVD to a rank determined by
                                       the unique singular values of input-file.
                                       Compares the sum of the sqaure roots of
                                       the squares of the singular values to
                                       SIMGA_THRESHOLD, which may be provided as
                                       an argument or defaults to 0.2.
                                       May not be used with -k

    -h, --help                       Show this message
        --version                    Show version
\end{verbatim}
\normalsize

\section{Examples}
For the following examples, you must have 
\begin{description}
\codeitem{svdimage in.jpg out.jpg -k 100}{Truncate \texttt{in.jpg} to have a rank of 100 and write it out as \texttt{out.jpg}. Ranks must be $1 \leq k < \min(\mbox{image\_height}, \mbox{image\_width})$.}
\codeitem{svdimage format1.png format2.gif -k 100}{Like previous, but notice how you can use file formats interchangeably.}
\codeitem{svdimage in.jpg out.jpg -a}{Truncate \texttt{in.jpg} to a ``reasonable'' rank. This algorithm attempts to remove unneeded ranks, but what ranks you consider unneeded may differ from me!}
\codeitem{svdimage in.jpg out.jpg -a 0.05}{Like previous, but with a specification of the sigma ratio. See the respective section for more information on how to use this.}
\codeitem{svdimage in.jpg out.svd -k 20}{Perform a truncation, but output to \texttt{.svd} format, a format designed to store SVDs.}
\codeitem{svdimage color.jpg bw.jpg -k 43 -c gray}{Decompose to grayscale. The default colorspace is rgb.}
\end{description}

\section{License}
Copyright (c) 2012 Tim Martin

Permission is hereby granted, free of charge, to any person obtaining
a copy of this software and associated documentation files (the
'Software'), to deal in the Software without restriction, including
without limitation the rights to use, copy, modify, merge, publish,
distribute, sublicense, and/or sell copies of the Software, and to
permit persons to whom the Software is furnished to do so, subject to
the following conditions:

The above copyright notice and this permission notice shall be
included in all copies or substantial portions of the Software.

THE SOFTWARE IS PROVIDED 'AS IS', WITHOUT WARRANTY OF ANY KIND,
EXPRESS OR IMPLIED, INCLUDING BUT NOT LIMITED TO THE WARRANTIES OF
MERCHANTABILITY, FITNESS FOR A PARTICULAR PURPOSE AND NONINFRINGEMENT.
IN NO EVENT SHALL THE AUTHORS OR COPYRIGHT HOLDERS BE LIABLE FOR ANY
CLAIM, DAMAGES OR OTHER LIABILITY, WHETHER IN AN ACTION OF CONTRACT,
TORT OR OTHERWISE, ARISING FROM, OUT OF OR IN CONNECTION WITH THE
SOFTWARE OR THE USE OR OTHER DEALINGS IN THE SOFTWARE.

\end{document}
